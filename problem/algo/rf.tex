\documentclass[11pt]{ctexart}  
\usepackage[top=2cm, bottom=2cm, left=2cm, right=2cm]{geometry}  
\usepackage{algorithm}  
\usepackage{algorithmicx}  
\usepackage{algpseudocode}  
\usepackage{amsmath}  
  
\floatname{algorithm}{算法}  
\renewcommand{\algorithmicrequire}{\textbf{输入:}}  
\renewcommand{\algorithmicensure}{\textbf{输出:}}  
  
\begin{document}  
    \begin{algorithm}  
        \begin{algorithmic}[1] %每行显示行号  
            \Require    $ntree$树的数目
            \Ensure 预测结果
            \Function{AntEpiSeeker}{$ntree$}
                \State 通过自助法(bootstrap)构建大小为n的一个训练集,即重复抽样选择n个训练样例
                \State 对于刚才新得到的训练集,构建一棵决策树tree
                \For{$i = 0 \to ntree$}  
                    \For{$i = 0 \to tree.nodes$}  
                        \State 通过不重复抽样选择d个特征
                        \State 利用上面的d个特征,选择某种度量分割节点
                    \EndFor
                \EndFor
                \State 对于每一个测试样例,对ntree颗决策树的预测结果进行投票。票数最多的结果就是随机森林的预测结果。
            \EndFunction  
        \end{algorithmic}  
    \end{algorithm}  
\end{document}  